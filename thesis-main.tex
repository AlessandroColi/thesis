\documentclass[12pt,a4paper,openright,twoside]{book}
\usepackage[italian,english]{babel}
\usepackage[utf8]{inputenc}
\usepackage{disi-thesis}
\usepackage{code-lstlistings}
\usepackage{notes}
\usepackage{shortcuts}
\usepackage{acronym}

\school{\unibo}
\programme{Corso di Laurea in Ingegneria e Scienze Informatiche}
\title{Kotlin Multiplatform e PulvReAKt: analisi e prototipazione di applicazioni IoT eterogenee} %davvero questo?
\author{Coli Alessandro}
\date{\today}
\subject{Programmazione ad Oggetti}
\supervisor{Prof. Viroli Mirko}
\cosupervisor{Dott. Farabegoli Nicolas}
\session{I}
\academicyear{2023-2024}

% Definition of acronyms
\acrodef{IoT}{Internet of Thing}
\acrodef{JVM}{Java Virtual Machine}
\acrodef{JS}{JavaScript}
\acrodef{MQTT}{Message Queuing Telemetry Transpor}
\acrodef{QoS}{Quality of Service}

\mainlinespacing{1.241} % line spacing in mainmatter, comment to default (1)

\begin{document}

\frontmatter\frontispiece

\begin{dedication}
dedica opzionale. Max a few lines.
\end{dedication}

%----------------------------------------------------------------------------------------
\tableofcontents   
%\listoffigures      (optional) comment if empty
%\lstlistoflistings (optional) comment if empty
%----------------------------------------------------------------------------------------

\mainmatter

%----------------------------------------------------------------------------------------
\chapter{Introduzione}
\label{chap:Introduzione}
%----------------------------------------------------------------------------------------
\paragraph{Structure of the Thesis}
\note{At the end, describe the structure of the paper}

\chapter{Background}
\label{chap:Background}

\section{Sviluppo Multiplatform}

Lo sviluppo multiplatform è diventato un approccio sempre più diffuso per la creazione di applicazioni che devono essere eseguite
su diverse piattaforme senza dover scrivere codice specifico per ciascuna di esse. Questo approccio consente agli sviluppatori di massimizzare
la condivisione del codice, riducendo i costi di sviluppo e manutenzione e garantendo una maggiore coerenza e compatibilità tra 
le diverse versioni dell'applicazione.

Esistono diversi approcci per lo sviluppo multiplatform, ciascuno con le proprie caratteristiche e vantaggi:
\begin{itemize}
\item \textbf{Codice Condiviso}: In questo approccio il codice condiviso viene scritto una volta e viene compilato per essere eseguito
 su diverse piattaforme. Questo può essere realizzato utilizzando linguaggi di programmazione che supportano la compilazione multiplatform, 
 come Kotlin, o utilizzando framework e librerie che consentono la condivisione del codice tra piattaforme diverse.
\item \textbf{Interfacce di Programmazione (API)}: Un altro approccio consiste nel definire un'API comune che fornisce un'interfaccia uniforme
 per l'accesso alle funzionalità dell'applicazione su diverse piattaforme. Questo consente di implementare la logica dell'applicazione specifica 
 per ciascuna piattaforma, mantenendo al contempo una coerenza nell'interfaccia e nei dati condivisi.
\item \textbf{Virtualizzazione e Contenitori}: Utilizzando tecnologie di virtualizzazione e contenitori, è possibile creare un ambiente uniforme 
per l'esecuzione dell'applicazione su diverse piattaforme. Questo approccio permette di isolare l'applicazione dall'infrastruttura sottostante, 
garantendo una maggiore portabilità e flessibilità.
\end{itemize}

Lo sviluppo multiplatform offre una serie di vantaggi significativi:
\begin{itemize}
\item \textbf{Massimizzazione della Condivisione del Codice}: Condividendo il codice tra le diverse piattaforme 
è possibile ridurre il lavoro ripetitivo e aumentare l'efficienza dello sviluppo.
\item \textbf{Riduzione dei Costi di Sviluppo}: Riducendo il numero di linee di codice da scrivere e gestire 
è possibile ridurre i costi complessivi di sviluppo e manutenzione dell'applicazione.
\item \textbf{Coerenza e Compatibilità}: Mantenendo una sola base di codice per tutte le versioni dell'applicazione 
è possibile garantire una maggiore coerenza e compatibilità tra le diverse versioni.
\item \textbf{Flessibilità}: Utilizzando approcci multiplatform gli sviluppatori possono adottare una varietà di tecnologie e 
linguaggi di programmazione per soddisfare le esigenze specifiche del progetto.
\end{itemize}

\section{Programmazione Distribuita}

Nel contesto dell'informatica moderna, la programmazione distribuita su piattaforme eterogenee emerge come un approccio fondamentale per affrontare le sfide
 della interconnessione in un ambiente eterogeneo di dispositivi e sistemi. Questo paradigma si basa sulla distribuzione delle responsabilità e delle risorse
 su una varietà di dispositivi e piattaforme, che possono differire per architettura hardware, sistema operativo e linguaggio di programmazione.

La programmazione distribuita su piattaforme eterogenee è guidata dalla necessità di integrare e coordinare dispositivi e sistemi diversi per creare soluzioni
 innovative e scalabili. Questo approccio richiede la collaborazione e la comunicazione efficace tra nodi distribuiti, che possono trovarsi in ambienti eterogenei
 come cloud, dispositivi embedded, server on-premise e dispositivi mobili.

Alcune tecnologie e approcci chiave utilizzati nella programmazione distribuita su piattaforme eterogenee includono:

\begin{itemize}
\item \textbf{Protocolli di Comunicazione Universali}: Questi protocolli consentono la comunicazione tra dispositivi e sistemi eterogenei,
 fornendo un'interfaccia standardizzata per lo scambio di dati e messaggi. Esempi di protocolli universali includono HTTP, \ac{MQTT} e gRPC,
  che consentono la comunicazione su diverse piattaforme e architetture.
\item \textbf{Middleware Distribuito}: Il middleware distribuito fornisce un livello di astrazione tra le applicazioni e l'infrastruttura sottostante,
 consentendo la trasparenza della distribuzione su piattaforme eterogenee. Questo può includere servizi di messaggistica, servizi di gestione delle transazioni
  e sistemi di caching distribuiti.
\item \textbf{Containerizzazione}: Le tecnologie di containerizzazione, come Docker e Kubernetes, consentono di confezionare,
 distribuire e gestire applicazioni su piattaforme eterogenee in modo uniforme. I container forniscono un'unità di distribuzione leggera e isolata,
  garantendo la portabilità e la scalabilità delle applicazioni su diverse infrastrutture.
\item \textbf{Orchestrazione Multi-Cloud}: L'orchestrazione multi-cloud permette di distribuire carichi di lavoro su più fornitori di servizi cloud,
 garantendo la ridondanza, la resilienza e la flessibilità delle applicazioni su piattaforme eterogenee
\end{itemize}

\section{Message Queuing Telemetry Transport (MQTT)}

Nel vasto panorama dell' \ac{IoT}, dove milioni di dispositivi sono interconnessi per scambiare dati e informazioni in tempo reale, 
\ac{MQTT}  si distingue come uno dei protocolli di comunicazione più importanti e ampiamente adottati. 
La sua storia affonda le radici nella necessità di comunicare in modo efficiente tra dispositivi con risorse limitate, come sensori,ù attuatori e dispositivi embedded.

\ac{MQTT} si presenta come una soluzione elegante per risolvere le sfide di comunicazione in un ambiente \ac{IoT}. 
Le sue caratteristiche distintive includono il basso utilizzo di banda e CPU, che lo rendono particolarmente adatto per i dispositivi con risorse limitate,
 e il modello di comunicazione publish-subscribe, che consente la trasmissione efficiente dei messaggi tra i partecipanti alla rete.

Il modello publish-subscribe di \ac{MQTT} è fondamentale per comprendere il suo funzionamento. In questo modello, i dispositivi si dividono in due ruoli principali:
 i publisher, che inviano i messaggi su specifici topic, e i subscriber, che si iscrivono a questi topic per ricevere i messaggi pertinenti. 
 Questo approccio consente una comunicazione flessibile e scalabile, in cui i dispositivi possono interagire in modo dinamico senza la necessità
 di conoscere direttamente gli indirizzi dei destinatari.

Un aspetto cruciale di \ac{MQTT} è la sua capacità di mantenere aperte le connessioni tra i dispositivi, riducendo al minimo la latenza e ottimizzando
 l'efficienza della comunicazione. Questo è particolarmente importante in contesti in cui è necessario uno scambio continuo di dati in tempo reale,
 come il monitoraggio ambientale, la gestione degli impianti industriali e la sorveglianza di sicurezza.

Inoltre, \ac{MQTT} offre funzionalità avanzate per la gestione delle connessioni, la sicurezza e la qualità del servizio (QoS),
 che consentono di personalizzare e ottimizzare la comunicazione in base alle esigenze specifiche dell'applicazione.
 Queste caratteristiche lo rendono un protocollo estremamente flessibile e adattabile, in grado di soddisfare una vasta gamma di requisiti
 e casi d'uso nell'ambito dell'\ac{IoT}.

\section{Dependency Injection}

Nel contesto dell'ingegneria del software, la \textit{Dependency Injection} rappresenta un pattern di progettazione fondamentale per la gestione
 delle dipendenze tra i vari componenti di un sistema. Questo approccio promuove un'architettura modulare e facilita la manutenzione, il testing e l'estensibilità del software.

La Dependency Injection è una tecnica attraverso la quale un oggetto (il \textit{client}) riceve le proprie dipendenze da un oggetto esterno 
(l' \textit{injector}), anziché crearle autonomamente. Le dipendenze sono componenti di cui il client ha bisogno per eseguire le proprie funzionalità.
 In altre parole, la Dependency Injection inverte il controllo della creazione delle dipendenze, trasferendo questa responsabilità a un container o a un framework.

Esistono tre principali varianti di Dependency Injection:
\begin{itemize}
\item \textbf{Constructor Injection}: Le dipendenze vengono fornite tramite il costruttore dell'oggetto. 
    Questo metodo garantisce che le dipendenze siano disponibili al momento della creazione dell'oggetto.
\item \textbf{Setter Injection}: Le dipendenze vengono fornite tramite metodi setter. 
    Questo approccio consente una maggiore flessibilità nella configurazione delle dipendenze dopo la creazione dell'oggetto.
\item \textbf{Interface Injection}: L'oggetto client espone un'interfaccia che permette al container di fornire le dipendenze. 
    Questo metodo è meno comune ma può essere utile in contesti specifici.
\end{itemize}

L'adozione della Dependency Injection offre numerosi vantaggi, tra cui:
\begin{itemize}
    \item \textbf{Maggiore modularità}: I componenti del sistema possono essere sviluppati e testati in isolamento, migliorando la coesione e riducendo l'accoppiamento.
    \item \textbf{Facilità di testing}: Le dipendenze possono essere sostituite con mock o stub, semplificando il processo di unit testing.
    \item \textbf{Manutenzione e estensibilità migliorate}: La separazione delle preoccupazioni facilita la modifica e l'estensione delle funzionalità del sistema
     senza impattare sul resto del codice.
    \item \textbf{Inversione del controllo}: La Dependency Injection implementa il principio di inversione del controllo,
     permettendo al framework di gestire il ciclo di vita degli oggetti e le loro dipendenze.
\end{itemize}

Diversi framework e container sono stati sviluppati per supportare la Dependency Injection in vari linguaggi di programmazione. Un esempio prominente, 
 in particolare per l'ecosistema Kotlin, è \textbf{Koin}: un framework di Dependency Injection per Kotlin che si distingue per la sua semplicità e leggerezza. 
 Koin utilizza una DSL (Domain Specific Language) per definire le dipendenze in modo chiaro e conciso, rendendo l'integrazione e la configurazione estremamente intuitive.
 Koin è particolarmente apprezzato nell'ambito dello sviluppo di applicazioni Android.

La Dependency Injection è un pattern di progettazione che offre una soluzione elegante alla gestione delle dipendenze in sistemi software complessi.
 La sua adozione consente di sviluppare applicazioni modulari, testabili e facilmente manutenibili, contribuendo a migliorare la qualità complessiva del software.
 Nelle sezioni successive, esploreremo più in dettaglio i vari aspetti della Dependency Injection, i suoi benefici e come implementarla efficacemente utilizzando diversi framework,
 con un focus particolare su Koin per applicazioni Kotlin.
    
 \section{PulvReAKt}
PulvReAKt è un framework leggero e multipiattaforma per Kotlin, progettato per facilitare la pulverizzazione dei sistemi, ovvero la suddivisione di applicazioni complesse
 in muyltemplici componenti modulari che possono essere distribuiti su diverse piattaforme. Il framework supporta \ac{JVM}, \ac{JS} e alcune piattaforme native, come Linux, macOS e iOS. 
 PulvReAKt consente di definire, configurare e distribuire componenti software utilizzando protocolli di comunicazione come \ac{MQTT} e RabbitMQ, rendendolo particolarmente adatto 
 per lo sviluppo di applicazioni distribuite e sistemi \ac{IoT}. Consente di suddividere l'applicazione in componenti indipendenti, facilitando lo sviluppo e la manutenzione. 
 Supporta \ac{JVM}, \ac{JS} e varie piattaforme native, permettendo l'esecuzione dei componenti in diversi ambienti.
 
\section{Kotlin}
La scelta del linguaggio utilizzato per questo progetto è stata relativamente semplice, poiché il progetto PulvReAKt, nel quale è contestualizzato, 
è scritto interamente in Kotlin con target Multiplatform. La maggior parte dello sviluppo è stata eseguita in Kotlin, con l'unica eccezione dell'utilizzo di una libreria 
JavaScript per lo sviluppo della parte \ac{JS}, integrata comunque all'interno di Kotlin grazie alla sua interoperabilità.

Kotlin è un linguaggio di programmazione moderno sviluppato da JetBrains, noto per il suo approccio pragmatico e la sua forte interoperabilità con Java. 
Dal suo lancio nel 2011, Kotlin ha guadagnato rapidamente popolarità, diventando nel 2017 il linguaggio ufficiale per lo sviluppo di applicazioni Android. 

Kotlin è stato progettato per essere conciso, riducendo la quantità di codice boilerplate necessario. Grazie alle funzioni lambda e alle estensioni di funzioni, 
il codice scritto in Kotlin risulta più leggibile e mantenibile rispetto a Java. Inoltre, Kotlin si propone di minimizzare gli errori a runtime, 
spostando la rilevazione degli errori al tempo di compilazione.

Uno dei punti di forza di Kotlin è la sua totale interoperabilità con Java. Questo significa che il codice Kotlin può chiamare ed essere chiamato da codice Java senza difficoltà. 
Questo facilita la migrazione graduale da Java a Kotlin nei progetti esistenti, permettendo agli sviluppatori di integrare Kotlin in maniera incrementale.

Kotlin introduce le coroutines, una potente astrazione per la programmazione asincrona e concorrente. Le coroutines offrono un modo semplice ed efficiente per gestire operazioni
che altrimenti richiederebbero la gestione complessa dei thread. Esse permettono di scrivere codice asincrono in uno stile sequenziale, migliorando la leggibilità e
riducendo il rischio di errori.

Le coroutines in Kotlin sono supportate a livello di linguaggio, con una libreria standard che fornisce molte funzionalità pronte all'uso. 
Ecco alcuni dei principali vantaggi delle coroutines:

\begin{itemize}
    \item \textbf{Semplicità del codice asincrono:} Le coroutines permettono di scrivere codice che sembra sincrono ma che viene eseguito in modo asincrono, 
    semplificando notevolmente la gestione delle operazioni I/O e altre operazioni bloccanti.
    \item \textbf{Efficienza delle risorse:} Le coroutines sono molto leggere rispetto ai thread tradizionali. Possono essere sospese e riprese senza un costo significativo, 
    consentendo una gestione efficiente delle risorse.
    \item \textbf{Facilità di cancellazione:} Le coroutines forniscono meccanismi per gestire la cancellazione delle operazioni in corso, 
    riducendo il rischio di risorse bloccate o perdite di memoria.
    \item \textbf{Composizione semplice:} Le coroutines permettono di comporre in modo semplice operazioni asincrone, rendendo il codice più modulare e facile da mantenere.
\end{itemize}

L'utilizzo delle coroutines in Kotlin rappresenta un approccio moderno alla programmazione concorrente, 
semplificando notevolmente il processo di scrittura e gestione del codice asincrono.

La sintassi concisa e le caratteristiche avanzate di Kotlin contribuiscono ad aumentare la produttività degli sviluppatori. 
Funzionalità come le funzioni di estensione, la gestione avanzata delle nullità e le espressioni lambda riducono il codice boilerplate e semplificano lo sviluppo, 
permettendo agli sviluppatori di concentrarsi sulla logica applicativa piuttosto che sui dettagli implementativi.

La forte tipizzazione e le funzionalità di sicurezza integrata di Kotlin contribuiscono a migliorare la qualità del codice. 
Il compilatore di Kotlin è in grado di rilevare molti tipi di errori a tempo di compilazione, riducendo il numero di bug presenti nel software finito. Inoltre, 
la gestione avanzata delle nullità e le funzioni di estensione aiutano a scrivere codice più robusto e meno incline agli errori.

Uno degli aspetti più potenti di Kotlin è la sua capacità di supportare lo sviluppo multiplatform. Grazie al quale gli sviluppatori possono scrivere 
un unico set di codice Kotlin che può essere condiviso e utilizzato su diverse piattaforme, come la \ac{JVM}, JavaScript e la compilazione nativa. 
Questo approccio consente agli sviluppatori di massimizzare la condivisione del codice, riducendo al minimo la duplicazione e semplificando la manutenzione dell'applicazione 
su più piattaforme.

\chapter{Requirements}
\label{chap:Requirements}

\section{Requisiti Funzionali}

\subsection{Scalabilità}
Il modulo di comunicazione deve supportare la scalabilità orizzontale, permettendo l'aggiunta di nuovi nodi senza interruzioni del servizio. Questo implica che l'architettura 
deve essere in grado di adattarsi dinamicamente all'aumento delle richieste, distribuendo il carico tra i nuovi nodi in modo efficiente. Una soluzione di scalabilità orizzontale 
efficace dovrebbe prevedere meccanismi di auto-scaling, dove i nuovi nodi vengono aggiunti automaticamente in risposta a picchi di carico e ridotti durante i periodi di 
bassa domanda, ottimizzando l'uso delle risorse e minimizzando i costi operativi. Deve inoltre essere in grado di gestire l'elasticità dell'infrastruttura, permettendo il 
provisioning e de-provisioning dei nodi in maniera automatizzata e senza interruzioni.

Dovendo gestire la comunicazione diretta tra molteplici componenti, si è scelto di utilizzare una struttura dei topic \ac{MQTT} per cui vi è un topic principale (PulvReAKt) 
al cui interno avvengono tutte le comunicazioni, secondo una struttura organizzata. Questa struttura è organizzata in modo gerarchico, seguendo lo schema 
PulvReAKt/mittente/destinatario. Per esempio, se un componente A volesse scrivere ad un componente B, il topic del messaggio sarebbe PulvReAKt/A/B. 
Questo approccio consente una gestione ordinata e tracciabile della comunicazione, facilitando la manutenzione e l'analisi dei flussi di messaggi all'interno del sistema distribuito.
Inoltre, questa struttura permette di filtrare e gestire facilmente i messaggi in base alla sorgente e alla destinazione, migliorando l'efficienza della comunicazione e 
riducendo il rischio di collisioni o confusione tra i messaggi inviati dai diversi componenti. La scelta di una struttura dei topic chiara e ben definita è cruciale per 
assicurare che la comunicazione tra i componenti rimanga scalabile e manutenibile man mano che il sistema cresce e si evolve.

\subsection{Affidabilità}
Un aspetto cruciale della scalabilità è la comunicazione tra i componenti, che deve essere robusta e affidabile per garantire il corretto funzionamento del sistema distribuito. 
I componenti devono poter inviare e ricevere messaggi utilizzando il protocollo di comunicazione \ac{MQTT}. Questo richiede l'integrazione di librerie specifiche per \ac{MQTT}, 
assicurando che i messaggi possano essere formattati, inviati e ricevuti correttamente. Deve essere garantita l'affidabilità nella consegna dei messaggi, assicurando che 
ogni messaggio venga ricevuto dal destinatario previsto. Ciò implica la gestione di conferme di ricezione, meccanismi di ritrasmissione in caso di fallimento e gestione delle 
code di messaggi per evitare perdite. Per raggiungere questo livello di affidabilità, è stato deciso di utilizzare il livello di Quality of Service (QoS) 2 di \ac{MQTT}. 
QoS 2 garantisce che un messaggio venga consegnato esattamente una volta al destinatario, richiedendo una sequenza di conferme di ricezione (acknowledgements) tra mittente e destinatario. 
Questo elimina il rischio di duplicati e garantisce la massima affidabilità, anche in condizioni di rete altamente instabili.

Un altro aspetto fondamentale è l'affidabilità del modulo di comunicazione, che deve garantire il funzionamento continuo e corretto del sistema anche in presenza di guasti o condizioni avverse. 
L'affidabilità può essere migliorata attraverso diverse strategie, tra cui la gestione delle connessioni e il monitoraggio delle prestazioni del sistema. 
È importante implementare meccanismi di rilevamento e recupero dagli errori per garantire che i messaggi non vengano persi o duplicati. Ad esempio, l'uso di sessioni persistenti 
permette di mantenere lo stato della connessione anche se il client o il broker si disconnettono temporaneamente. In caso di disconnessione, i messaggi pubblicati con QoS 2 
vengono memorizzati e consegnati non appena la connessione viene ristabilita, assicurando che nessun messaggio importante venga perso.

Inoltre, il modulo dovrebbe includere funzionalità di logging e monitoraggio per tracciare l'attività e le performance del sistema. Questi strumenti permettono di identificare 
e risolvere rapidamente eventuali problemi, migliorando la reattività del sistema alle condizioni di errore. Il logging dettagliato degli eventi di comunicazione aiuta 
anche nella diagnosi e nella prevenzione di guasti futuri, fornendo dati preziosi per l'analisi delle cause principali dei problemi.

\section{Interoperabilità}

L'interoperabilità si riferisce alla capacità dei diversi sistemi e componenti software di lavorare insieme senza problemi, scambiando dati e utilizzando le informazioni scambiate 
in modo efficace. Nel contesto di PulvReAKt, questo significa che i componenti sviluppati per \ac{JVM} devono poter interagire con quelli sviluppati in JavaScript o con quelli 
che girano su piattaforme native. Questo richiede che il protocollo di comunicazione, in questo caso \ac{MQTT}, sia implementato in modo standard e conforme su tutte le piattaforme.

\subsection{Java Virtual Machine}

La \ac{JVM} è una macchina virtuale che esegue bytecode Java, ed è la piattaforma di riferimento per lo sviluppo in Java e Kotlin. 
Essa offre una serie di caratteristiche che la rendono una scelta popolare per l'implementazione di applicazioni complesse e ad alte prestazioni.
Le caratteristiche principali della \ac{JVM} sono:

\begin{itemize}
\item \textbf{Interoperabilità}: La \ac{JVM} permette l'interoperabilità tra Kotlin e Java, consentendo l'uso di librerie e strumenti Java consolidati, 
 permettendo agli sviluppatori di sfruttare l'ampio ecosistema Java esistente.
\item \textbf{Portabilità}: Il bytecode generato può essere eseguito su qualsiasi dispositivo dotato di una \ac{JVM}, garantendo portabilità cross-platform. 
Questo è particolarmente utile per le applicazioni distribuite su una vasta gamma di dispositivi e sistemi operativi.
\item \textbf{Ottimizzazioni di Runtime}: La \ac{JVM} include un Just-In-Time (JIT) compiler che ottimizza il codice durante l'esecuzione, migliorando le performance e 
consentendo un'esecuzione più veloce delle applicazioni Kotlin rispetto a linguaggi interpretati o compilati staticamente.
\item \textbf{Garbage Collection}: Gestione automatica della memoria tramite garbage collection, riducendo il rischio di memory leaks. 
Allevia gli sviluppatori dalla necessità di gestire manualmente l'allocazione e la deallocazione della memoria, semplificando lo sviluppo e riducendo il rischio di errori.
\end{itemize}

\subsection{JavaScript e Node.js}

JavaScript è il linguaggio di scripting dominante per lo sviluppo web client-side, eseguito all'interno dei browser. Kotlin/\ac{JS} permette di compilare codice Kotlin in JavaScript, 
sfruttando le capacità di questa piattaforma. Le caratteristiche principali di \ac{JS} sono:

\begin{itemize}
\item \textbf{Compatibilità con il Web}: JavaScript è nativamente supportato dai browser, permettendo l'esecuzione di applicazioni Kotlin/\ac{JS} direttamente nel contesto web e 
consente agli sviluppatori di creare applicazioni web interattive e dinamiche utilizzando Kotlin come linguaggio di sviluppo.
\item \textbf{Ecossistema di Librerie}: Ampia disponibilità di librerie e framework per lo sviluppo web, che possono essere utilizzati anche con Kotlin/\ac{JS}. 
Include librerie per la gestione del DOM, l'interazione con API REST, la manipolazione dei dati JSON e molto altro ancora, consentendo agli sviluppatori di sfruttare 
le funzionalità esistenti senza dover reinventare la ruota.
\item \textbf{Asincronia}: Supporto nativo per le operazioni asincrone tramite callback, Promises e async/await che vanno a sopperire alla mancanza di capacità multithread 
essendo progettato per lavorare anche in contesto browser, quindi con un singolo thread. Questa particolarita è estremamente utile per le applicazioni web che devono gestire 
operazioni di rete, come il caricamento di dati da un server remoto.
\item \textbf{Interoperabilità}: Kotlin/\ac{JS} permette l'interoperabilità con le API JavaScript esistenti, facilitando l'integrazione con altre tecnologie web,
consentendo agli sviluppatori di utilizzare librerie e framework JavaScript esistenti all'interno del codice Kotlin, con una maggiore flessibilità 
e libertà di scelta nella progettazione delle applicazioni.
\end{itemize}

\begin{itemize}
\item \textbf{Node.js} rappresenta una potente piattaforma per lo sviluppo backend, estendendo le capacità di JavaScript oltre il browser e offrendo un ambiente efficiente e scalabile 
per la creazione di applicazioni server-side moderne. Utilizzando Node.js, gli sviluppatori possono beneficiare di un ecosistema ricco di strumenti avanzati per costruire
applicazioni robuste e performanti.
\item \textbf{Event-driven e Non-blocking I/O}: Node.js utilizza un modello event-driven e non-blocking per le operazioni di I/O, consentendo di gestire un numero elevato di connessioni
 concorrenti con un singolo thread grazie al meccanismo delle promise in \ac{JS}.
\item \textbf{Ecosistema di Moduli }: Node.js ha un vasto ecosistema di moduli e pacchetti disponibili tramite npm (Node Package Manager). 
Questi moduli coprono una vasta gamma di funzionalità, come la gestione del database, la manipolazione dei file e la creazione di server wen. 
\end{itemize}

\subsection{Native}

Kotlin/Native permette di compilare codice Kotlin direttamente in codice macchina eseguibile su dispositivi target, senza la necessità di una macchina virtuale o un interprete.
Le caratteristcihe principale delle piattaforme native sono:

\begin{itemize}
\item \textbf{Esecuzione Diretta}: Compilazione in codice nativo che viene eseguito direttamente dall'hardware, senza layer di interpretazione. 
Che si traduce in prestazioni superiori rispetto alle applicazioni eseguite su una macchina virtuale o interprete, poiché non vi è alcun overhead dovuto alla traduzione 
del bytecode in istruzioni di macchina.
\item \textbf{Performance}: Le applicazioni native tendono ad avere performance superiori rispetto a quelle eseguite su una macchina virtuale o interprete, 
grazie alla compilazione diretta. Questo è particolarmente importante per le applicazioni ad alte prestazioni che richiedono un tempo di risposta rapido e una bassa latenza.
\item \textbf{Accesso a Funzionalità di Sistema}: Possibilità di accedere direttamente alle API di sistema e alle risorse hardware, offrendo maggior controllo e ottimizzazione. 
Consentendo agli sviluppatori di creare applicazioni che interagiscono direttamente con l'hardware sottostante, sfruttando appieno le capacità del dispositivo target.
\item \textbf{Assenza di Overhead}: Nessun overhead dovuto alla gestione della macchina virtuale o alla garbage collection, sebbene Kotlin/Native includa un garbage collector. 
Questo significa che le risorse del sistema possono essere utilizzate in modo più efficiente, riducendo il consumo di memoria e CPU.
\item \textbf{Cross-platform}: Supporto per diverse piattaforme, inclusi iOS, Windows, Linux e macOS, facilitando lo sviluppo di applicazioni multi-piattaforma. 
Consentendo agli sviluppatori di scrivere un'unica base di codice Kotlin che può essere compilata per eseguire su una varietà di dispositivi e sistemi operativi, 
riducendo al minimo lo sforzo di sviluppo e la complessità del codice.
\end{itemize}

\chapter{Design}
\label{chap:Design}

\section{Package}

\section{Struttura}

\section{Logica dei Test}

\section{Tecnologie Utilizzate}
In questo capitolo, esploreremo le tecnologie e le librerie utilizzate nel progetto, analizzando le loro caratteristiche e i loro ruoli.

\subsection{Kotest}
Kotest rappresenta una scelta chiave per l'ambiente di sviluppo, poiché offre un framework di testing multipiattaforma specificamente progettato per Kotlin. 
Questo framework, oltre a semplificare il processo di scrittura e di esecuzione dei test, si distingue per la sua flessibilità e le funzionalità avanzate. 
Grazie al supporto per test di proprietà, test parametrizzati e altre caratteristiche, Kotest garantisce prestazioni eccellenti e affidabilità nei risultati, 
fornendo un solido fondamento per la verifica e la convalida del codice Kotlin.

\subsection{Arrow}
Arrow è una libreria potente che porta la programmazione funzionale idiomatica in Kotlin, arricchendo il linguaggio con concetti come i tipi di dati immutabili, le monadi e le frecce.
L'integrazione di Arrow nel progetto migliora significativamente la qualità del codice, facilitando la manutenzione e rendendo il codice più leggibile, testabile 
e meno soggetto a errori. Questa libreria contribuisce in modo tangibile alla costruzione di applicazioni Kotlin più robuste e manutenibili.

\subsection{Kmqtt}

Kmqtt riveste un ruolo cruciale nell'implementazione della comunicazione tramite \ac{MQTT} nel progetto. Questa libreria, progettata per Kotlin Multiplatform, 
offre un'implementazione unificata del client \ac{MQTT} su diverse piattaforme, dalla \ac{JVM} al Native. La sua facilità d'uso e la flessibilità consentono di integrare la comunicazione
 \ac{MQTT} in modo uniforme e affidabile su tutte le piattaforme supportate, fornendo una solida base per lo sviluppo delle funzionalità di comunicazione nell'applicazione.

\subsection{MQTT.js}

MQTT.js è una libreria essenziale per lo sviluppo di applicazioni web che richiedono comunicazione \ac{MQTT} direttamente in ambienti JavaScript. Grazie alla sua semplicità d'uso
 e alle prestazioni elevate, \ac{MQTT}.js si è affermata come una scelta popolare tra gli sviluppatori che lavorano nell'ambito delle applicazioni web interattive. 
 La sua integrazione nell'ecosistema di sviluppo permette di implementare in modo efficace e affidabile la comunicazione \ac{MQTT} nei progetti JavaScript.

\subsection{Mosquitto}

Mosquitto rappresenta il cuore dell'infrastruttura \ac{MQTT} del progetto, offrendo un broker \ac{MQTT} open source che supporta sia l'hosting locale che il servizio pubblico. 
La leggerezza e la configurabilità di Mosquitto lo rendono una scelta ideale per creare un ambiente di sviluppo e testing robusto per le applicazioni basate su \ac{MQTT}. 
Le sue funzionalità avanzate, come l'autenticazione e la sicurezza delle connessioni, garantiscono un'esperienza di sviluppo sicura e affidabile.

\subsection{Git}

Git svolge un ruolo fondamentale nel controllo di versione del codice sorgente del progetto. Questo sistema di controllo di versione distribuito consente agli sviluppatori 
di tracciare le modifiche al codice, collaborare in modo efficiente con altri membri del team e gestire diverse versioni del progetto. La sua flessibilità e la vasta gamma 
di funzionalità supportate lo rendono una scelta versatile per il controllo di versione in progetti di qualsiasi dimensione e complessità.

\subsection{Gradle}

Gradle rappresenta il cuore del processo di build e automazione del progetto. Questo sistema di automazione della compilazione, con la sua configurabilità ed estendibilità, 
semplifica la gestione delle dipendenze, la compilazione del codice e l'esecuzione dei test. Grazie alla sua integrazione con Kotlin, Gradle contribuisce in modo significativo 
a rendere il processo di sviluppo più efficiente e organizzato, offrendo un ambiente di sviluppo ottimizzato e scalabile.


\chapter{Implementazione e Validazione}
\label{chap:Implementazione e Validazione}

\section{Sviluppo}

\section{Validazione}

\section{Prestazione sulle diverse piattaforme}

\section{Integrazione in PulvReAKt}

\chapter{Conclusioni}
\label{chap:Conclusioni}

%----------------------------------------------------------------------------------------
% BIBLIOGRAPHY
%----------------------------------------------------------------------------------------

\backmatter

\nocite{*} % Remove this as soon as you have the first citation

\bibliographystyle{alpha}
\bibliography{bibliography}

\begin{acknowledgements} % this is optional
Optional. Max 1 page.
\end{acknowledgements}

\end{document}
